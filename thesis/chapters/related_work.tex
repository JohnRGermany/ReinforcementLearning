\chapter{Related Work}
\label{section:related_work}

This chapter will shortly present the most important breakthroughs and influences for this thesis in recent years.


RL approaches have gained a lot of attention in the last few years.
Starting with DeepMind's 2013 paper \emph{Playing Atari with Deep Reinforcement Learning} \cite{atari} a new interest in this field arose not only in the public but also in science.
Following DeepMind's original work, many extensions were presented, some of them with high and others with not so much impact.
The most prominent additions to the DQN released were \emph{Deep Reinforcement Learning with Double Q-learning} \cite{ddqn}, \emph{Prioritized Experience Replay} \cite{per} and \emph{Dueling Network Architectures for Deep Reinforcement Learning} \cite{duel_dqn} which are all covered in this thesis.
Exploration in these algorithms was mainly achieved by naively choosing random actions in some time steps.


Following DQN deep methods for continuous action spaces evolved in the form of \emph{Deep Deterministic Policy Gradient} in the paper \emph{Continuous control with deep reinforcement learning} \cite{ddpg}.
At first exploration was achieved here by adding noise to the predicted actions, either correlated in the form of an \emph{Ornstein–Uhlenbeck process} or uncorrelated in the form of \emph{uniform sampling}.
Plappert et al presented a new method to apply noise not to the action space but directly to the parameters of the network \cite{param_noise}.
The problem here is that this is only useful for off-policy algorithms such as DDPG because on-policy directly learn from the policy, meaning noise added to it disturbs learning.
A method for on-policy algorithms was presented but it was shown that it did not affect the continuous controls environments we used in this thesis.


After DDPG new stochastic algorithms were presented that achieved very good results in continuous control tasks such as the \emph{Swimmer-v1} environment.
The most prominent examples are \emph{Trust Region Policy Optimization} \cite{trpo} and \emph{Proximal Policy Optimization Algorithms} \cite{ppo}.
These algorithms have built in exploration because they act stochastically rather than deterministically.


So far this thesis gave an introduction in RL as it is today, it gave a general overview of the theoretical background and it presented similar work of others in this direction.
In the next chapter we will present the implementation of the algorithms used in this thesis.
